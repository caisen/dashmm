\chapter{Advanced Guide to DASHMM}
\label{ch:advanced}

In this chapter, the complete interface to DASHMM will be presented. This will
cover some repeat information as the previous chapter, but will provide more
details that are not relevant for the basic usage of DASHMM. In a few cases,
the coverage of a library construct is complete in the previous chapter, and
so those topics will not be repeated here.

The fundamental difference between the basic and advanced interface to
DASHMM is that the advanced interface is needed when a user is implementing
a new Expansion or Method. This will require a user to be familiar with more
of the details of how the execution is performed, and will thus require some
more details about how HPX-5 provides parallelism to DASHMM. However, the
extent to which a user will have to learn HPX-5 directly is still extremely
limited.

The arrangement of material in this chapter will parallel the arrangement in
the previous.

\section{DASHMM Concepts}

This sections covers the conceptual framwork of both DASHMM's implementation
of general multipole methods and the parallelization of those methods with
HPX-5.

\subsection{Multipole method abstractions}

Both mathematically, and in the code. So the operations and the template
arguments.

\subsection{Parallelization abstractions}

More detail about this that is relevant for advanced use of the library.


\section{Basic types}

Domain Geometry. Index. ExpansionRole. Operation.

\section{Initializing DASHMM}

Go over some of the relevant hpx related arguments here for init.

\section{Evaluation}

Basically just add the distribution policy stuff.

\section{DASHMM arrays}

Do we need the alternate constructor? From the hpx address?

\section{Array map actions}

Go over decomposition template parameter.

\section{DAG objects}

\section{Built-in methods}

\section{Built-in expansions}

\section{Built-in distribution policies}

Also go over the default policy

\section{User-defined Expansions}

\section{User-defined Methods}

\section{User-defined distribution policies}
