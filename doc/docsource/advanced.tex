\chapter{Advanced Guide to DASHMM}
\label{ch:advanced}

In this chapter, the complete interface to DASHMM will be presented. This will
cover some repeat information as the previous chapter, but will provide more
details that are not relevant for the basic usage of DASHMM. In a few cases,
the coverage of a library construct is complete in the previous chapter, and
so those topics will not be repeated here.

The fundamental difference between the basic and advanced interface to
DASHMM is that the advanced interface is needed when a user is implementing
a new Expansion or Method. This will require a user to be familiar with more
of the details of how the execution is performed, and will thus require some
more details about how HPX-5 provides parallelism to DASHMM. However, the
extent to which a user will have to learn HPX-5 directly is still extremely
limited.

The arrangement of material in this chapter will parallel the arrangement in
the previous.

\section{DASHMM Concepts}

This sections covers the conceptual framwork of both DASHMM's implementation
of general multipole methods and the parallelization of those methods with
HPX-5.

\subsection{Multipole method abstractions}

Both mathematically, and in the code. So the operations and the template
arguments.

\subsection{Parallelization abstractions}

More detail about this that is relevant for advanced use of the library.



\section{Basic types}

The following basic types are needed when implementing user-defined methods and
expansions.

\subsection{\texttt{Operation}}

DASHMM organizes the transformations between various forms of the potential
expansion into a set of operations. The \texttt{Operation} scoped enumeration
identifies the operations that DASHMM recognizes. They are: \texttt{Nop},
\texttt{StoM}, \texttt{StoL}, \texttt{MtoM}, \texttt{MtoL}, \texttt{LtoL},
\texttt{MtoT}, \texttt{LtoT}, \texttt{StoT}, \texttt{MtoI}, \texttt{ItoI}
and \texttt{ItoL}.

\subsection{\texttt{ExpansionRole}}

DASHMM creates Expansion objects in a number of roles. The role is essentially
a description of which tree in the dual tree an expansion is most closely
associated with, and whether the expansion is a primary or intermediate
expansion.  The \texttt{ExpansionRole} is an enumeration with the following
members: \texttt{kSourcePrimary}, \texttt{kSourceIntermediate},
\texttt{kTargetPrimary}, \texttt{kTargetIntermediate} and
\texttt{kNoRoleNeeded}. The latter is for situations that that require an
operation from the Expansion type, but which do not require expansion data.
The prototypical example of this is the \texttt{S->T} operation.

\subsection{\texttt{Index}}

Each node of both the source and target trees can be identified with four
integers: the level of the tree (starting with 0 for the root), and the
position of the low corner of the node in units of the node size at the given
level. The index allows DASHMM to not only perform accurate positional
comparisions, but also to provide an ordering of the nodes of the tree at a
given level.

The \texttt{Index} type has the following members:

\begin{itemize}
  \item \texttt{Index::Index(int ix = 0, int iy = 0, int iz = 0, int lvl = 0)}:
    Construct an Index with the given on-level position and level.
  \item \texttt{int Index::x() const}: Return the on-level position in the
    x direction.
  \item \texttt{int Index::y() const}: Return the on-level position in the
    y direction.
  \item \texttt{int Index::z() const}: Return the on-level position in the
    z direction.
  \item \texttt{int Index::level() const}: Return the level of the index.
  \item \texttt{Index Index::parent(int num = 1) const}: Return the
    \texttt{num}-th parent of this \texttt{Index}.
  \item \texttt{Index Index::child(int which) const}: Return the given child
    of this Index. \texttt{which} is a composite value that indicates with
    each bit if the child is on the left or right in that direction. For
    example for \texttt{which == 6}, the child is the left child in the
    z-direction, and the right child in both the y and x directions.
  \item \texttt{int Index::which\_child() const}: Return which child of its
    parent, this node is.
  \item \texttt{bool Index::operator==(const Index \&other)}: Equality operator.
\end{itemize}

\subsection{\texttt{DomainGeometry}}

To represent the computational domain of the sources and targets, DASHMM
uses the \texttt{DomainGeometry} object. This object can be used to convert
an \texttt{Index} into \texttt{Point}s for various locations in the volume
represented by that index. The domains represented by DASHMM are cubical
regions.

The \texttt{DomainGeometry} object provides the following methods:

\begin{itemize}
  \item \texttt{DomainGeometry::DomainGeometry()}: Default construct a domain
    to begin at the origin, with a zero side length.
  \item \texttt{DomainGeometry::DomainGeometry(Point low, double size)}:
    construct a domain with the given low corner, and the given side length.
  \item \texttt{DomainGeometry::DomainGeometry(Point low, Point high,
    double f = 1.0)}: Construct a domain from the given low and high corners.
    The given region need not be cubical. The resulting object will represent
    the smallest cube that contains the specified volume. Additionally, the
    final parameter can be used to enlarge the resulting cube by a fixed
    fraction.
  \item \texttt{double DomainGeometry::size() const}: return the side length
    of the represented cubical volume.
  \item \texttt{Point DomainGeometry::low() const}: return the low corner of
    the represented volume.
  \item \texttt{Point DomainGeometry::high() const}: return the high corner of
    the represented volume.
  \item \texttt{Point DomainGeometry::center() const}: return the center of the
    represented volume.
  \item \texttt{Point DomainGeometry::low\_from\_index(Index idx) const}: return
    the low corner of the region represented by the given index.
  \item \texttt{Point DomainGeometry::high\_from\_index(Index idx) const}: return
    the high corner of the region represented by the given index.
  \item \texttt{Point DomainGeometry::center\_from\_index(Index idx) const}: return
    the center of the region represented by the given index.
  \item \texttt{double DomainGeometry::size\_from\_index(Index idx) const}: return
    the size of the region represented by the given index.
\end{itemize}



\section{Initializing DASHMM}

Initialization of DASHMM via \texttt{init()} requires providing the command
line arguments to the program. This is to provide the opportunity to HPX-5 to
detect any command line arguments that modify its behavior. A full description
of the available options can be found in the HPX-5 documentation. Here, we
shall cover those that are most relevant for programs using DASHMM.

NOTE: The material in this section should not be considered to be part of the
library's interface, and are subject to modification out of the control of
the DASHMM development team.

The following HPX-5 command line arguments are the most relevant to DASHMM:

\begin{itemize}
  \item \texttt{--hpx-help}: Display a help message giving a brief description
    of all available options.
  \item \texttt{--hpx-threads}: Specify the number of scheduler threads that
    HPX-5 will use per rank. Typically, one thread per core gives best results,
    but fewer is sometimes useful in scalability studies.
  \item \texttt{--hpx-heapsize}: Specify the size in bytes of the amount of
    global address space available to each rank. The default is frequently too
    low for large problem sizes. This, however, should not be set to take all
    of the system memory.
\end{itemize}



\section{Evaluation}

The \texttt{Evaluator} object has one additional feature that was not
covered in the previous chapter. In addition to the parameters outlined
in Chapter~\ref{ch:basic}, one final optional parameter is available.

Each Method has a distribution policy that specifies how the DAG nodes are
to be distributed around the available resources. To allow for these policies
to have some parameters selectable at runtime, the
\texttt{Evaluator::evaluate} method accepts a distribution policy object.
This allows for users to fine-tune the behavior of the distribution if needed.
When the final argument is not supplied, DASHMM employs a default constructed
object of the type \texttt{Method<Source, Target, Expansion>::distropolicy\_t},
where \texttt{Method, Source, Target} and \texttt{Expansion} are the template
arguments supplied to the particular instance of \texttt{Evaluator}.



\section{Array map actions}

The \texttt{ArrayMapAction} utility class supports a third template parameter
that controls the level of parallelism employed in the mapping of the work
to the records in the array. This parameter is an integer that has the
following meaning: if the argument is zero, the array will be handled in a
single chunk; if the argument is positive, the array will be handled in a number
of chunks equal to the number of HPX-5 scheduler threads times the provided
argument.

The default value of this argument is 1, so the default operation of
the map will be to split each rank's portion of the array into a number of
equally sized pieces equal to the number of scheduler threads. Unless there is
the possibility for variation in the amount of computation that the mapped
action performs per record, it is likely that the default value will be
the sufficient. If, however, the work for each record is variable, better
performance may be achieved with smaller chunks, and thus larger third arguments
to the template.

The full declaration of \texttt{ArrayMapAction} is as follows:

\begin{verbatim}
template <typename T, typename E, int factor = 1>
class ArrayMapAction;
\end{verbatim}



\section{DAG objects}

After the dual tree is constructed, DASHMM creates an explicit representation
of the DAG for the given method applied to the just constructed tree. This DAG
is used to perform a work distribution, and to act as a scaffold from which
the actual expansion data is instantiated. For users wishing to implement
their own methods, the DAG objects are the objects that will be needed. In
DASHMM, the Method builds a DAG from the dual tree. There are 4 classes that
make up the DAG system for DASHMM: \texttt{DAG}, \texttt{DAGInfo},
\texttt{DAGNode} and \texttt{DAGEdge}, which will be covered in turn.

\subsection{\texttt{DAGEdge}}

The edges connecting nodes in the DAG is described with the \texttt{DAGEdge}
type. This simple type holds a pointer to the source and target
\texttt{DAGNode} connected by the edges, the \texttt{Operation} that the edge
represents, and an integer weight that gives an estimate of the communication
cost of the edge. The weight is optionally used by the distribution policy to
aid in the decision about data placement around the system. The full definition
of \texttt{DAGEdge} is as follows:

\begin{verbatim}
struct DAGEdge {
  DAGNode *source;          /// Source node of the edge
  DAGNode *target;          /// Target node of the edge
  Operation op;             /// Operation to perform along edge
  int weight;               /// estimate of communication cost if it occurs

  DAGEdge() : source{nullptr}, target{nullptr}, op{Operation::Nop}, weight{0} {}
  DAGEdge(DAGNode *start, DAGNode *end, Operation inop, int w)
    : source{start}, target{end}, op{inop}, weight{w} {}
};
\end{verbatim}

\subsection{\texttt{DAGNode}}

The nodes of the DAG are represented by the simple type \texttt{DAGNode}. It
contains the following public members:

\begin{itemize}
  \item \texttt{std::vector<DAGEdge> out_edges}: the edges of the DAG that
    start at this node.
  \item \texttt{std::vector<DAGEdge> in_edges}: the edges of the DAG that end
    at this node.
  \item \texttt{Index idx}: the index of the tree node to which this DAG node
    is associated.
  \item \texttt{int locality}: the locality to which this node will be assigned
    by the distribution. Note that this will most often be set by the
    distribution policy. To indicate that the locality is not set, a value of
    \texttt{-1} should be used.
  \item \texttt{int color}: A color that might be used by the distribution
    policy. This has meaning only in the context of the distribution policy.
\end{itemize}

\noindent Additionally, there are three member functions:\

\begin{itemize}
  \item \texttt{DAGNode::DAGNode(Index i)}: Construct a DAG node. This will set
    the index to the given value, give the locality a value of \texttt{-1}, and
    default construct the remaining members of the node.
  \item \texttt{void DAGNode::add_out_edge(DAGNode *end, Operation op,
    int weight)}: Add an out edge from this node to the given node with the
    given operation and the given weight. Note, this will not add the equivalent
    edge to the target node of the edge.
  \item \texttt{void DAGNode::add_in_edge(DAGNode *start, Operation op,
    int weight)}: Add an in edge from the given node to this node with the
    given operation and the given weight. Note, this will not add the equivalent
    edge to the source node of the edge.
\end{itemize}

\noindent In practice, the latter two are typically not needed, see
\texttt{DAGInfo} below.


\subsection{\texttt{DAG}}

The \texttt{DAG} object is given to a distribution policy when computing the
localities of the nodes in the DAG. This is another simple object that contains
four containers of the nodes of the DAG. These containers separate the nodes
by their position in the DAG.

The \texttt{DAG} object has the following members:

\begin{itemize}
  \item \texttt{std::vector<DAGNode *> source_leaves}: Nodes of the DAG that
    are associated with leaves of the source tree. These nodes will have no
    incoming edges.
  \item \texttt{std::vector<DAGNode *> source_nodes}: All other DAG nodes that
    are associated with nodes of the source tree.
  \item \texttt{std::vector<DAGNode *> target_leaves}: Nodes of the DAG that are
    associated with leaves of the target tree. These nodes will have no outgoing
    edges. In some methods, these are not associated with the leaves of the
    target tree, but rather those nodes of the target tree after which more
    refinement for the method would only induce unnecessary overhead.
  \item \texttt{std::vector<DAGNode *> target_nodes}: All other DAG nodes that
    are associated with node of the target tree.
\end{itemize}


\subsection{\texttt{DAGInfo}}

The \texttt{DAGInfo} object contains all the information related to the DAG
for each node of the source and target trees. These objects will be the
primary means by which a Method interacts with the DAG.


\section{Built-in methods}



\section{Built-in expansions}



\section{Built-in distribution policies}

Also go over the default policy



\section{User-defined Expansions}



\section{User-defined Methods}



\section{User-defined distribution policies}
