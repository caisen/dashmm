\chapter{Advanced Guide to DASHMM}
\label{ch:advanced}

In this chapter, the rest of the interface to DASHMM will be presented. Some
of the material presented here provides further information on constructs
presented in the previous chapter. In other cases, the coverage of a library
construct is complete in the previous chapter, and so those topics will not be
repeated here. Finally, some constructs not needed for basic DASHMM usage
are presented in their entirety

The fundamental difference between the basic and advanced interface to
DASHMM is that the advanced interface is needed when a user is implementing
a new Expansion or Method. This will require a user to be familiar with more
of the details of how the execution is performed, and will thus require some
more details about how HPX-5 provides parallelism to DASHMM. However, the
extent to which a user will have to learn HPX-5 directly is still extremely
limited.

The arrangement of material in this chapter will parallel the arrangement in
the previous.

\section{DASHMM Concepts}
\label{sec:advconcepts}

This sections covers the conceptual framework of DASHMM's implementation
of general multipole methods. The following abstractions form the basis for the
implementation of multipole methods in DASHMM.

\subsection{The Dual Tree}

The multipole method framework implemented in DASHMM is general enough to allow
for situations where the sources of an interaction, and the locations where the
interaction is to be computed are different. As a result, DASHMM uses the Dual
Tree construction. This involves a hierarchical partitioning of the problem
domain for both the source and target points. This leads naturally to two
trees, one each for sources and targets. These tree are constructed to be
compatible with one another: the root of the source and target trees represent
the same volume. Each node of both trees can have up to eight children,
each having half the side length of the parent.

When referring to particular volumes in the tree, DASHMM often uses an integer
index giving which level of the tree and where on that level the node is. This
makes it possible to refer to the volumes of the tree nodes without needing to
wrangle with floating point arithmetic.

When these trees are constructed, the source and target points are ultimately
assigned to a node of the associated tree. The refinement of the tree is
adaptive in DASHMM; a node is refined, adding up to eight children, until the
number of points in a node is below some given threshold. In this way, the
tree adapts to the distribution of source and target points.

\subsection{The DAG}

Ultimately, a multipole method computation is the creation and evaluation of
a directed acyclic graph (DAG). This DAG encodes in its nodes the various
approximations, or expansions, that are used to the effects of the sources
represented by a given node. Encoded in the edges of the DAG are the operations
that transform the various approximations into other approximations.

DASHMM constructs two versions of the DAG representing a given multipole
computation: the first is an explicit representation that is used during the
discovery of the overall structure of the DAG and which is used to distribute
the work among the available localities; the second is an implicit
representation that lives in the data, synchronization objects, and the
interconnections among the tasks that are performed in the actual computation.

The explicit DAG is represented in DASHMM with objects that an advanced user of
DASHMM might need to interact with. The implicit DAG comprises a number of
HPX-5-aware objects that manage the parallel execution. The exact use and form
of these objects is not necessary for even an advanced user; DASHMM insulates
the user from these details.

\subsection{Expansion Roles}

Each node of the DAG is associated with a node of either the source or target
tree. This leads to the four expansion roles defined in DASHMM. Nodes
associated with the typical expansions (multipole and local) are termed
`normal'. Nodes associated with expansions that are used in advanced techniques
such as merge-and-shift are called `intermediate' expansions. This leads to
the four roles for expansions: two kinds for each of two trees.

\subsection{Operations}

The abstraction of the multipole method in DASHMM allows for the following
operations. Of course, for a particular use-case, these operations may take
on meanings different from the typical meaning for FMM or BH. The notation for
the following operations all take the form \texttt{AtoB}, where \texttt{A} is
one of \texttt{\{S, M, L, I\}} and \texttt{B} is one of the
following \texttt{\{M, L, I, T\}}. \texttt{S} is a node of the DAG that
represents the leaves of a source tree; it represents a set of sources.
\texttt{M} is a node of the DAG representing a normal expansion in the source
tree. \texttt{L} is a node of the DAG representing a normal expansion in the
target tree. \texttt{T} is a node of the DAG representing a leaf of the
target tree; it represents a set of target locations. \texttt{I} represents
the intermediate expansions on either the source or target trees (no
distinction is made in this notation as the source or target side of the
\texttt{I} can be inferred from the operation).

The \texttt{StoM} operation is any operation that produces an approximation
or summary version of the source data in a given node. These operations
generate multipole moments from sources in typical use cases.

The \texttt{MtoM} operation is used to combine summaries of a set of sources
into summaries that apply to larger volumes, or to larger number of sources.
Typical uses of this operation produce a hierarchy of multipole expansions
for every node of the source tree.

The \texttt{MtoL} operation is used to translate summaries of a set of sources
into a summary that applies to a volume in the target tree. This is the
operation that distinguishes FMM from BH; the creation of the local expansions
allows for efficient evaluation of large parts of the source distribution.

The \texttt{LtoL} operation is used to translate a local expansion that applies
for a parent into a local expansion that applies in the child's volume.

The \texttt{MtoT} operation computes the effect of a multipole expansion on
a set of targets.

The \texttt{LtoT} operation computes the effect of a local expansion on a
set of targets.

The \texttt{StoT} operation computes the direct effect of a set of sources on
a set of targets. If a method only schedules \texttt{StoT} operations, it will
be the same as the direct summation method.

The \texttt{StoL} operation computes the effect of a set of sources on a
volume of the target tree.

The \texttt{MtoI} operation transforms the summary information on the source
tree into another form that certain advanced methods might find useful. The use
of these intermediate expansions is what distinguishes FMM and FMM97. The
intermediate expansion is associated with a source tree node.

The \texttt{ItoI} operation translates the intermediate representation on the
source tree into an intermediate representation on the target tree. The forms
of these representations may well be different.

The \texttt{ItoL} operation translates the intermediate representation on the
target tree into a local expansion on the target tree.

\subsection{Expansions and Views}

There is a distinction between an Expansion in the mathematical sense, and the
Expansions in DASHMM. An Expansion in DASHMM might contain multiple mathematical
expansions. It can be the case that a given method might need to keep
several versions of summary data for a given set of sources. In this case,
a DASHMM expansion presents multiple Views. Each View is a single mathematical
expansion. The various views that are implemented in an Expansion can be
related to one another, but it can also support the ability to have unrelated
mathematical expansions to allow for a single evaluation to compute for
multiple kernels at the same time.

\subsection{Kernel details}

The mathematical concept of a Kernel is included in DASHMM's concept of an
Expansion. Ultimately the Method does not care about how the operations are
performed, just that they exist. So it is the Expansion that implements the
kernel. In the most direct sense, the kernel can be seen in the implemented
\texttt{StoT} operation.

Some kernels will have parameters that control the exact shape of the potential.
For example, the built-in Yukawa kernel is scaling variant, and so the
scale must be provided to DASHMM. These kernel parameters also impact the
expansions that are employed in Expansion. Further, often there is some one-time
precomputation that can occur that produces values that are used in the various
operations. To allow for this precomputation, DASHMM employs the abstraction
of a Kernel table. This table is created once before the operations are employed
and is made available to all ranks. The table represents any values needed
by the expansions and operations that does not change as the inputs to the
operations change.

\subsection{Methods and DAG creation}

There are four main operations that a Method will employ during DAG discovery.

The first is \emph{generate}. This operation is responsible for creating an
approximation at the leaves of a source tree from the source records. This is
often the generation of the multipole expansions from the source data.

The second is \emph{aggregate}. In aggregate, the summary expansions computed
at the child of a given node are combined into a summary that corresponds to
all of the sources that are descendants of the given node.

The third is \emph{inherit}. The inherit operation will take summary
information for target nodes and will create summary information for the
children of that node. Note that operations that end at a target leaf are not
discovered in this operation.\footnote{This is an asymmetry that will be
rectified in future versions of DASHMM.}

The fourth is \emph{process}. It is in process that the bulk of the work of
creating the DAG is done. This is responsible for forging any connection
between the two trees. Notice that generate and aggregate only connect nodes
on the source tree, and that inherit only connects nodes in the target tree.
All other connections are made in process.


\section{Basic types}

The following basic types are needed when implementing user-defined methods and
expansions.

\subsection{\texttt{Operation}}

DASHMM organizes the transformations between various forms of the potential
expansion into a set of operations. The \texttt{Operation} scoped enumeration
identifies the operations that DASHMM recognizes. They are: \texttt{Nop},
\texttt{StoM}, \texttt{StoL}, \texttt{MtoM}, \texttt{MtoL}, \texttt{LtoL},
\texttt{MtoT}, \texttt{LtoT}, \texttt{StoT}, \texttt{MtoI}, \texttt{ItoI}
and \texttt{ItoL}.

\subsection{\texttt{ExpansionRole}}

DASHMM creates Expansion objects in a number of roles. The role is essentially
a description of which tree in the dual tree an expansion is most closely
associated with, and whether the expansion is a primary or intermediate
expansion.  The \texttt{ExpansionRole} is an enumeration with the following
members: \texttt{kSourcePrimary}, \texttt{kSourceIntermediate},
\texttt{kTargetPrimary}, \texttt{kTargetIntermediate} and
\texttt{kNoRoleNeeded}. The latter is for situations that that require an
operation from the Expansion type, but which do not require expansion data.
The prototypical example of this is \texttt{StoT}.

\subsection{\texttt{Index}}

Each node of both the source and target trees can be identified with four
integers: the level of the tree (starting with 0 for the root), and the
position of the low corner of the node in units of the node size at the given
level. The index allows DASHMM to not only perform accurate positional
comparisons, but also to provide an ordering of the nodes of the tree at a
given level.

\begin{lstlisting}
Index::Index(int ix = 0, int iy = 0, int iz = 0, int lvl = 0)
\end{lstlisting}

\noindent Construct an Index with the given on-level position and level.

\begin{lstlisting}
int Index::x() const
\end{lstlisting}

\noindent Return the on-level position in the x direction.

\begin{lstlisting}
int Index::y() const
\end{lstlisting}

\noindent Return the on-level position in the y direction.

\begin{lstlisting}
int Index::z() const
\end{lstlisting}

\noindent Return the on-level position in the z direction.

\begin{lstlisting}
int Index::level() const
\end{lstlisting}

\noindent Return the level of the index.

\begin{lstlisting}
Index Index::parent(int num = 1) const
\end{lstlisting}

\noindent Return the \texttt{num}-th parent of this \texttt{Index}. By default,
this gives the index of the immediate parent.

\begin{lstlisting}
Index Index::child(int which) const
\end{lstlisting}

\noindent Return the given child of this Index. \texttt{which} is a
composite value
that indicates with each bit if the child is on the left or right in that
direction. For example for \texttt{which == 6}, the child is the left child
in the z-direction, and the right child in both the y and x directions.

\begin{lstlisting}
int Index::which_child() const
\end{lstlisting}

\noindent Return which child of its parent this node is.

\begin{lstlisting}
bool Index::operator==(const Index &other) const
\end{lstlisting}

\noindent Equality operator.


\subsection{\texttt{DomainGeometry}}

To represent the computational domain of the sources and targets, DASHMM
uses \texttt{DomainGeometry} object. This object can be used to convert
an \texttt{Index} into \texttt{Point}s for various locations in the volume
represented by that index. The domains represented by DASHMM are cubical
regions.

\begin{lstlisting}
DomainGeometry::DomainGeometry()
\end{lstlisting}

\noindent Default construct a domain to begin at the origin, with a zero
side length.

\begin{lstlisting}
DomainGeometry::DomainGeometry(Point low, double size)
\end{lstlisting}

\noindent Construct a domain with the given low corner, and the given side
length.

\begin{lstlisting}
DomainGeometry::DomainGeometry(Point low, Point high, double f = 1.0)
\end{lstlisting}

\noindent Construct a domain from the given low and high corners. The given
region need not be cubical. The resulting object will represent the smallest
cube that contains the specified volume. Additionally, the final parameter can
be used to enlarge the resulting cube by a fixed fraction.

\begin{lstlisting}
double DomainGeometry::size() const
\end{lstlisting}

\noindent Return the side length of the represented cubical volume.

\begin{lstlisting}
Point DomainGeometry::low() const
\end{lstlisting}

\noindent Return the low corner of the represented volume.

\begin{lstlisting}
Point DomainGeometry::high() const
\end{lstlisting}

\noindent Return the high corner of the represented volume.\\

\begin{lstlisting}
Point DomainGeometry::center() const
\end{lstlisting}

\noindent Return the center of the represented volume.

\begin{lstlisting}
Point DomainGeometry::low_from_index(Index idx) const
\end{lstlisting}

\noindent Return the low corner of the region represented by the given index.

\begin{lstlisting}
Point DomainGeometry::high_from_index(Index idx) const
\end{lstlisting}

\noindent Return the high corner of the region represented by the given index.

\begin{lstlisting}
Point DomainGeometry::center_from_index(Index idx) const
\end{lstlisting}

\noindent Return the center of the region represented by the given index.

\begin{lstlisting}
double DomainGeometry::size_from_index(Index idx) const
\end{lstlisting}

\noindent Return the size of the region represented by the given index.



\section{Initializing DASHMM}

Initialization of DASHMM via \texttt{init()} requires providing the command
line arguments to the program. This is to provide the opportunity to HPX-5 to
detect any command line arguments that modify its behavior. A full description
of the available options can be found in the HPX-5 documentation. Here, we
shall cover those that are most relevant for programs using DASHMM.

NOTE: The material in this section should not be considered to be part of the
library's interface, and are subject to modification out of the control of
the DASHMM development team.

The following HPX-5 command line arguments are the most relevant to DASHMM:

\begin{lstlisting}
--hpx-help
\end{lstlisting}

\noindent Display a help message giving a brief description of all
available options.

\begin{lstlisting}
--hpx-threads
\end{lstlisting}

\noindent Specify the number of scheduler threads that HPX-5 will use per rank.
Typically, one thread per core gives best results, but fewer is sometimes
useful in scalability studies.

\begin{lstlisting}
--hpx-heapsize
\end{lstlisting}

\noindent Specify the size in bytes of the amount of global address space
available to each rank. The default is frequently too low for large problem
sizes. This, however, should not be set to take all of the system memory.



\section{Evaluation}

\subsection{Distribution Policy}

The \texttt{Evaluator} object has one additional feature that was not
covered in the previous chapter. In addition to the parameters outlined
in Chapter~\ref{ch:basic}, one final optional parameter is available.

Each Method has a distribution policy that specifies how the DAG nodes are
to be distributed around the available resources. To allow for these policies
to have some parameters selectable at runtime, the
\texttt{Evaluator::evaluate} method accepts a distribution policy object.
This allows for users to fine-tune the behavior of the distribution if needed.
When the final argument is not supplied, DASHMM employs a default constructed
object of the type \texttt{Method<Source, Target, Expansion>::distropolicy\_t},
where \texttt{Method, Source, Target} and \texttt{Expansion} are the template
arguments supplied to the particular instance of \texttt{Evaluator}.

\subsection{Expanded API}

In addition to the one-size-fits-all approach implemented in the
\texttt{evaluate()}, \texttt{Evaluator} objects expose sub-steps of the
evaluation process. This is to enable uses in iterative methods, where the
same DAG is used multiple times. Rather than creating the Dual Tree and
discovering the DAG each iteration, they can be created only once, and
reused.

\begin{lstlisting}
DualTreeHandle Evaluator::create_tree(
    const Array<Source> &sources,
    const Array<Target> &targets,
    int refinement_limit)
\end{lstlisting}

\noindent Create a Dual Tree object in the global address space from the
given \texttt{sources} and \texttt{targets}, and with the given
\texttt{refinement\_limit}. The return value of this method is an opaque
handle to the created Dual Tree. This value, being a handle, can be freely
copied.

This is a collective call and all ranks must participate.

\begin{lstlisting}
std::unique_ptr<DAG> Evaluator::create_DAG(
    DualTreeHandle tree,
    int n_digits,
    const std::vector<double> *kernel_params,
    const Method<Source, Target, Expansion<Source, Target>> *method,
    distropolicy_t distro = distropolicy_t{})
\end{lstlisting}

\noindent Given a \texttt{tree}, an accuracy parameter \texttt{n\_digits},
the kernel
parameters, \texttt{kernel\_params} to use for this DAG, an instance of a
Method object, and an optional distribution policy instance, this method will
create the DAG corresponding to the given method for the given Dual Tree.

This is a collective call, and all ranks must participate.

\begin{lstlisting}
ReturnCode Evaluator::execute_DAG(DualTreeHandle tree, DAG *dag)
\end{lstlisting}

\noindent Once the DAG has been constructed, this method will execute the work
represented by the given DAG. This will return either \texttt{kSuccess} on
successful execution, or \texttt{kRuntimeError} if there is a problem during
the execution.

After this call, the DAG is left in a used state. If the DAG is to be used
again, it must be reset (see \texttt{reset\_DAG} below).

This is a collective call, and all ranks must participate.

\begin{lstlisting}
ReturnCode Evaluator::reset_DAG(DAG *dag)
\end{lstlisting}

\noindent This will take a DAG that has been executed with the
\texttt{execute\_DAG}
method, and will reset the internal data structures so that it might be used
again. Note that this does not actually perform the execution of that DAG,
merely prepares it to be executed.

This will return either \texttt{kSuccess} if there is no error, or
\texttt{kRuntimeError} in the case of an error.

This is a collective call, and all ranks must participate.

\begin{lstlisting}
ReturnCode Evaluator::destroy_DAG(
    DualTreeHandle tree, 
    std::unique_ptr<DAG> dag)
\end{lstlisting}

\noindent This will destroy a DAG. This will return either \texttt{kSuccess}
if the destruction proceeds without error, or will return
\texttt{kRuntimeError}.

This is a collective call, and all ranks must participate.

\begin{lstlisting}
ReturnCode Evaluator::destroy_tree(DualTreeHandle tree)
\end{lstlisting}

\noindent This method destroys the Dual Tree specified by \texttt{tree}. This
will return either \texttt{kSuccess} after successfully destroying the tree, or
\texttt{kRuntimeError} if there is some error.

This is a collective call, and all ranks must participate.


\section{Serializer}
\label{sec:serializer}

Source and Target data may at times need to be copied around the system, either
locally or across the network. To allow for non-trivial Source and Target
types, DASHMM allows the user to specify a serialization manager. This is
a subclass of the \texttt{Serializer} abstract class, which has the job of
correctly serializing and deserializing the data in a Source or Target
record. Each \texttt{Array} has a serialization manager associated with it that
takes responsibility for this serialization.

\subsection{\texttt{TrivialSerializer}}

DASHMM comes with one built in subclass of \texttt{Serializer}, called
\texttt{TrivialSerializer}, which is actually a template over a single
template parameter. It can work for any trivially-copyable type, and is the
default serialization manager installed when a new \texttt{Array} is
allocated.

\subsection{\texttt{Serializer} Interface}

The interface that all subclasses of \texttt{Serializer} must implement is
given below.

\begin{lstlisting}
size_t Serializer::size(void *object) const
\end{lstlisting}

\noindent This method will return the serialized size in bytes of the given
object. The interface does not contain any information about the type of the
object, and so it is up to the user of DASHMM to assure that the subclass is
used only for compatible objects. Typically, inside the implementation, the
provided address will be typecast.

\begin{lstlisting}
void *Serializer::serialize(void *object, void *buffer) const
\end{lstlisting}

\noindent This method will serialize the given object into the provided 
\texttt{buffer}. This method then returns the address \texttt{buffer + bytes},
where \texttt{bytes} is the number of bytes of the serialized form of 
\texttt{object}.

\begin{lstlisting}
void *Serializer::deserialize(void *buffer, void *object) const
\end{lstlisting}

\noindent This method will deserialize an object from \texttt{buffer} into
the provided \texttt{object}. This object already exists; this method fills
in the correct values for the object, rather than creating a new object.
The return value from this method is \texttt{buffer + bytes} where 
\texttt{bytes} is the serialized size of the object.

\section{Array}

When an \texttt{Array} contains records of a non-trivial type, the user of
the array must provide an object that derives from \texttt{Serializer}, which
has the task of serialization of those records. Without explicitly setting the
serialization manager, an \texttt{Array} will initially have a
\texttt{TrivialSerializer<T>} as its serialization manager, where \texttt{T} 
is the record type of the array.

At any point, an \texttt{Array}'s serialization manager can be changed with 
the following method of \texttt{Array}.

\begin{lstlisting}
ReturnCode Array<T>::set_manager(std::unique_ptr<Serializer> manager)
\end{lstlisting} 

\noindent This method will set the serialization manager for the array to the 
provided \texttt{Serializer} instance. This is a collective call, and all ranks
must participate. Further, though it would be technically possible to do 
otherwise, each rank should provide an object of the same subclass of 
\texttt{Serializer}.

\noindent

\section{Array For Each Actions}

The \texttt{ArrayForEachAction} utility class supports a third template
parameter that controls the level of parallelism employed in the mapping of
the work to the records in the array. This parameter is an integer that has the
following meaning: if the argument is zero, the array will be handled in a
single chunk; if the argument is positive, the array will be handled in a
number of chunks equal to the number of HPX-5 scheduler threads times the
provided argument.

The default value of this argument is 1, so the default operation of
the map will be to split each rank's portion of the array into a number of
equally sized pieces equal to the number of scheduler threads. Unless there is
the possibility for variation in the amount of computation that the mapped
action performs per record, it is likely that the default value will be
sufficient. If, however, the work for each record is variable, better
performance may be achieved with smaller chunks, and thus larger third
arguments to the template.

The full declaration of \texttt{ArrayForEachAction} is as follows:

\begin{lstlisting}[frame=]
template <typename T, typename E, int factor = 1>
class ArrayForEachAction;
\end{lstlisting}



\section{DAG objects}

After the dual tree is constructed, DASHMM creates an explicit representation
of the DAG for the given method applied to the just constructed tree. This DAG
is used to perform a work distribution, and to act as a scaffold from which
the actual expansion data is instantiated. For users wishing to implement
their own methods, the DAG objects are the objects that will be needed. In
DASHMM, the Method builds a DAG from the dual tree. There are 4 classes that
make up the DAG system for DASHMM: \texttt{DAG}, \texttt{DAGInfo},
\texttt{DAGNode} and \texttt{DAGEdge}, which will be covered in turn.

\subsection{\texttt{DAGEdge}}

The edges connecting nodes in the DAG is described with the \texttt{DAGEdge}
type. This simple type holds pointers to the source and target
\texttt{DAGNode}s connected by the edge, the \texttt{Operation} that the edge
represents, and an integer \texttt{weight} that gives an estimate of the
communication cost of the edge. The \texttt{weight} is optionally used by the
distribution policy to aid in the decision about data placement around the
system. The full definition of \texttt{DAGEdge} is as follows:

\begin{lstlisting}
DAGNode *DAGEdge::target
\end{lstlisting}

\noindent Target node of the edge.

\begin{lstlisting}
Operation DAGEdge::op
\end{lstlisting}

\noindent Operation to perform along edge.

\begin{lstlisting}
int DAGEdge::weight
\end{lstlisting}

\noindent Estimate of communication cost required if the edge were to span
localities.


\subsection{\texttt{DAGNode}}

The nodes of the DAG are represented by the simple type \texttt{DAGNode}. It
contains the following public members:

\begin{lstlisting}
std::vector<DAGEdge> DAGNode::out_edges
\end{lstlisting}

\noindent The edges of the DAG that start at this node.

\begin{lstlisting}
int DAGNode::locality
\end{lstlisting}

\noindent The locality to which this node will be assigned by the
distribution. Note that this will most often be set by the distribution policy.
To indicate that the locality is not set, a value of \texttt{-1} should be used.

\begin{lstlisting}
int DAGNode::color
\end{lstlisting}

\noindent A color that might be used by the distribution policy. This has
meaning only in the context of the distribution policy.

\begin{lstlisting}
DAGNode::DAGNode(Index i)
\end{lstlisting}

\noindent Construct a DAG node. This will set the index to the given value,
give the locality a value of \texttt{-1}, and default construct the remaining
members of the node.

\begin{lstlisting}
size_t DAGNode::in_count() const
\end{lstlisting}

\noindent Returns the number of incoming edges to this node.

\begin{lstlisting}
size_t DAGNode::out_count() const
\end{lstlisting}

\noindent Returns the number of outgoing edges from this node.

\begin{lstlisting}
Index DAGNode::index()
\end{lstlisting}

\noindent Returns the index of the tree node to which this DAG node is
associated.

\begin{lstlisting}
bool DAGNode::is_parts() const
\end{lstlisting}

\noindent Indicates if this node represents sources or targets.

\begin{lstlisting}
bool DAGNode::is_normal() const
\end{lstlisting}

\noindent Indicates if this node represents a multipole or local expansion.

\begin{lstlisting}
bool DAGNode::is_interm() const
\end{lstlisting}

\noindent Indicates if this node represents an intermediate expansion.


\subsection{\texttt{DAG}}

The \texttt{DAG} object is given to a distribution policy when computing the
localities of the nodes in the DAG. This is another simple object that contains
four containers of the nodes of the DAG. These containers separate the nodes
by their position in the DAG.

The \texttt{DAG} object has the following members:

\begin{lstlisting}
std::vector<DAGNode *> DAG::source_leaves
\end{lstlisting}

\noindent Nodes of the DAG that are associated with leaves of the source tree.
These nodes will have no incoming edges.

\begin{lstlisting}
std::vector<DAGNode *> DAG::source_nodes
\end{lstlisting}

\noindent All other DAG nodes that are associated with nodes of the source tree.

\begin{lstlisting}
std::vector<DAGNode *> DAG::target_leaves
\end{lstlisting}

\noindent Nodes of the DAG that are associated with leaves of the target tree.
These nodes will have no outgoing edges. In some methods, these are not
associated with the leaves of the target tree, but rather those nodes of the
target tree after which more refinement for the method would only induce
unnecessary overhead.

\begin{lstlisting}
std::vector<DAGNode *> DAG::target_nodes
\end{lstlisting}

\noindent All other DAG nodes that are associated with node of the target tree.



\subsection{\texttt{DAGInfo}}

The \texttt{DAGInfo} object contains all the information related to the DAG
for each node of the source and target trees. These objects will be the
primary means by which a Method interacts with the DAG.

Each \texttt{DAGInfo} object represents up to three DAG nodes: a particle
node, the normal DAG node, and an intermediate DAG node. The particle node
represents the terminal nodes of the DAG: either the information about
the sources, or the information about the targets. These DAG nodes do not
represent expansions. The normal
DAG node represents expansions with a role of \texttt{kSourcePrimary} or
\texttt{kTargetPrimary}. The intermediate node represents expansions with a
role of \texttt{kSourceIntermediate} or \texttt{kTargetIntermediate}. A
particular method might not use certain expansion roles, so a
\texttt{DAGInfo} object might not contain the normal or intermediate DAG node.
Further, not all tree nodes are leaves, and so only some \texttt{DAGInfo}
objects will have a particles node.

This object manages the concurrent modification of the DAG. However, that
management is hidden from users of this object.

The \texttt{DAGInfo} methods that a user might need to use in a Method are
covered below.

\begin{lstlisting}
Index DAGInfo::index() const
\end{lstlisting}

\noindent Return the index of the volume represented by the DAG nodes owned by
this object.

\begin{lstlisting}
bool DAGInfo::add_normal()
\end{lstlisting}

\noindent Add a normal DAG node to this object. This will return \texttt{true}
if this call added the node. Otherwise, if the \texttt{DAGNode} already owns a
normal node, this call will do nothing, and return false.

\begin{lstlisting}
bool DAGInfo::add_interm()
\end{lstlisting}

\noindent Add an intermediate DAG node to this object. This will return
\texttt{true} if this call added the node. Otherwise, if the \texttt{DAGNode}
already owns an intermediate node, this call will do nothing, and return
\texttt{false}.

\begin{lstlisting}
bool DAGInfo::add_parts()
\end{lstlisting}

\noindent Add a particles DAG node to this object. This will return
\texttt{true} if this call added the node. Otherwise, if the \texttt{DAGNode}
already owns a particles node, this call will do nothing, and return false.

\begin{lstlisting}
bool DAGInfo::has_normal() const
\end{lstlisting}

\noindent Predicate returning if this object owns a normal DAG node.

\begin{lstlisting}
bool DAGInfo::has_interm() const
\end{lstlisting}

\noindent Predicate returning if this object owns an intermediate DAG node.

\begin{lstlisting}
bool DAGInfo::has_parts() const
\end{lstlisting}

\noindent Predicate returning of this object owns a particles DAG node.

\begin{lstlisting}
void DAGInfo::set_normal_locality(int loc)
\end{lstlisting}

\noindent Set the locality for the normal DAG node, if it exists, to the given
locality, \texttt{loc}.

\begin{lstlisting}
void DAGInfo::set_interm_locality(int loc)
\end{lstlisting}

\noindent Set the locality for the intermediate DAG node, if it exists, to
the given locality, \texttt{loc}.

\begin{lstlisting}
void DAGInfo::StoM(DAGInfo *source, int weight)
\end{lstlisting}

\noindent Connect the particles DAG node of \texttt{source} to the normal DAG
node of this object with operation \texttt{StoM}, and assign the resulting
edge the given \texttt{weight}. It is an error to call this method if this
object does not own a normal DAG node, and if the \texttt{source} does not
own a particles DAG node.

\begin{lstlisting}
void DAGInfo::StoL(DAGInfo *source, int weight)
\end{lstlisting}

\noindent Connect the particles DAG node of \texttt{source} to the normal DAG
node of this object with operation \texttt{StoL}, and assign the resulting
edge the given \texttt{weight}. It is an error to call this method if this
object does not own a normal DAG node, and if the \texttt{source} does not
own a particles DAG node.

\begin{lstlisting}
void DAGInfo::MtoM(DAGInfo *source, int weight)
\end{lstlisting}

\noindent Connect the normal DAG node of \texttt{source} to the normal DAG
node of this object with operation \texttt{MtoM}, and assign the resulting
edge the given \texttt{weight}. It is an error to call this method if this
object does not own a normal DAG node, and if the \texttt{source} does not
own a normal DAG node.

\begin{lstlisting}
void DAGInfo::MtoL(DAGInfo *source, int weight)
\end{lstlisting}

\noindent Connect the normal DAG node of \texttt{source} to the normal DAG
node of this object with operation \texttt{MtoL}, and assign the resulting
edge the given \texttt{weight}. It is an error to call this method if this
object does not own a normal DAG node, and if the \texttt{source} does not
own a normal DAG node.

\begin{lstlisting}
void DAGInfo::LtoL(DAGInfo *source, int weight)
\end{lstlisting}

\noindent Connect the normal DAG node of \texttt{source} to the normal DAG
node of this object with operation \texttt{LtoL}, and assign the resulting
edge the given \texttt{weight}. It is an error to call this method if this
object does not own a normal DAG node, and if the \texttt{source} does not
own a normal DAG node.

\begin{lstlisting}
void DAGInfo::MtoT(DAGInfo *target, int weight)
\end{lstlisting}

\noindent Connect the normal DAG node of this object to the particles DAG
node of \texttt{target} with operation \texttt{MtoT}, and assign the
resulting edge the given \texttt{weight}. It is an error to call this method
if this object does not own a normal DAG node, and if the \texttt{target}
does not own a particles DAG node.

\begin{lstlisting}
void DAGInfo::LtoT(DAGInfo *target, int weight)
\end{lstlisting}

\noindent Connect the normal DAG node of this object to the particles DAG
node of \texttt{target} with operation \texttt{LtoT}, and assign the
resulting edge the given \texttt{weight}. It is an error to call this method
if this object does not own a normal DAG node, and if the \texttt{target}
does not own a particles DAG node.

\begin{lstlisting}
void DAGInfo::StoT(DAGInfo *source, int weight)
\end{lstlisting}

\noindent Connect the particles DAG node of \texttt{source} to the particles
DAG node of this object with operation \texttt{StoT}, and assign the
resulting edge the given \texttt{weight}. It is an error to call this method
if this object does not own a particles DAG node, and if the \texttt{source}
does not own a particles DAG node.

\begin{lstlisting}
void DAGInfo::MtoI(DAGInfo *source, int weight)
\end{lstlisting}

\noindent Connect the normal DAG node of \texttt{source} to the intermediate
DAG node of this object with operation \texttt{MtoI}, and assign the
resulting edge the given \texttt{weight}. It is an error to call this method
if this object does not own an intermediate DAG node, and if the
\texttt{source} does not own a normal DAG node.

\begin{lstlisting}
void DAGInfo::ItoI(DAGInfo *source, int weight)
\end{lstlisting}

\noindent Connect the intermediate DAG node of \texttt{source} to the
intermediate DAG node of this object with operation \texttt{ItoI}, and
assign the resulting edge the given \texttt{weight}. It is an error to call
this method if this object does not own an intermediate DAG node, and if the
\texttt{source} does not own an intermediate DAG node.

\begin{lstlisting}
void DAGInfo::ItoL(DAGInfo *source, int weight)
\end{lstlisting}

\noindent Connect the intermediate DAG node of \texttt{source} to the normal
DAG node of this object with operation \texttt{ItoL} operation, and assign
the resulting edge the given \texttt{weight}. It is an error to call this
method if this object does not own a normal DAG node, and if the
\texttt{source} does not own an intermediate DAG node.



\section{\texttt{ViewSet}}

The \texttt{ViewSet} object is a means by which a subset of the possible views
of an Expansion can be selected. The DASHMM concept of Expansion allows for
multiple mathematical expansions to be contained in the Expansion object. This
is a feature used by some advanced methods. A \texttt{ViewSet} essentially
indexes over the possible views of an Expansion. For example, if an
Expansion has six versions (or six views) of the data, then a \texttt{ViewSet}
might select a subset of these (e.g., views 2, 4 and 5). Like anything in C++,
the indices are zero based. The particular meaning of each index is defined
only with respect to the Expansion's implementation.

\begin{lstlisting}
ViewSet::ViewSet()
\end{lstlisting}

\noindent Default construct the object.

\begin{lstlisting}
ViewSet::ViewSet(
    ExpansionRole role,
    const Point &center = Point{},
    double scale = 1.0)
\end{lstlisting}

\noindent Create an empty \texttt{ViewSet}, while setting a few vital features:
the \texttt{role} of the represented expansion, the \texttt{center} around
which it is defined, and the scaling factor \texttt{scale} associated with the
represented expansion.

\begin{lstlisting}
void ViewSet::clear()
\end{lstlisting}

\noindent Clear out the object. After \texttt{clear()}, it will be as if the
object were default constructed.

\begin{lstlisting}
void ViewSet::add_view(int index)
\end{lstlisting}

\noindent Add a view with the given \texttt{index}. This is an incomplete view.

\begin{lstlisting}
void ViewSet::add_view(int index, size_t bytes, char *data)
\end{lstlisting}

\noindent Add a view with the given \texttt{index}, size in \texttt{bytes} and
\texttt{data}. This is a complete view, indicating not only which index, but
also the data backing the view.

NOTE: The \texttt{ViewSet} does not assume ownership of \texttt{data}.

\begin{lstlisting}
void ViewSet::set_bytes(int view, size_t bytes)
\end{lstlisting}

\noindent Set the size of a given view. \texttt{view} gives which view of this
object to set the size of. That is, \texttt{view} is not an index into the
original Expansion, but an index into the set of views represented in this
object.

\begin{lstlisting}
void ViewSet::set_data(int view, char *data)
\end{lstlisting}

\noindent Set the data for a given view. \texttt{view} gives which view of this
object to set the size of. That is, \texttt{view} is not an index into the
original Expansion, but an index into the set of views represented in this
object.

\begin{lstlisting}
void ViewSet::set_role(ExpansionRole role)
\end{lstlisting}

\noindent Set the \texttt{role} of this object.

\begin{lstlisting}
void ViewSet::set_center(const Point &center)
\end{lstlisting}

\noindent Set the \texttt{center} of this object. This represents the point
around which the expansion is defined.

\begin{lstlisting}
void ViewSet::set_scale(double s)
\end{lstlisting}

\noindent Set the \texttt{scale} of this object.

\begin{lstlisting}
int ViewSet::view_index(int view) const
\end{lstlisting}

\noindent Return the index in the original Expansion that is represented by the
given \texttt{view}.

\begin{lstlisting}
size_t ViewSet::view_bytes(int view) const
\end{lstlisting}

\noindent Return the size in bytes of the given \texttt{view}.

\begin{lstlisting}
char *ViewSet::view_data(int view) const
\end{lstlisting}

\noindent Return the data of the given \texttt{view}.

\begin{lstlisting}
ExpansionRole ViewSet::role() const
\end{lstlisting}

\noindent Return the role of the represented expansion.

\begin{lstlisting}
Point ViewSet::center() const
\end{lstlisting}

\noindent Return the center of the represented expansion.

\begin{lstlisting}
double ViewSet::scale() const
\end{lstlisting}

\noindent Return the scale of the represented expansion.

\begin{lstlisting}
int ViewSet::count() const
\end{lstlisting}

\noindent Return the number of views in this object.

\begin{lstlisting}
size_t ViewSet::bytes() const
\end{lstlisting}

\noindent Return the total size of the data represented by this object. This
includes not only the data of the views themselves, but some metadata used
during serialization of this object.


\section{Tree nodes}

The nodes of the dual tree are implemented as the class \texttt{Node<T>}, which
is a template over the record type, either Source or Target. Typically these
types are aliased to \texttt{sourcenode\_t = Node<Source>}, and
\texttt{targetnode\_t = Node<Target>}. The members of this class that might be
needed by implementers of new Methods are detailed below.

\begin{lstlisting}
Index Node<T>::idx
\end{lstlisting}

\noindent The index of the tree node.

\begin{lstlisting}
Node<T> *Node<T>::parent
\end{lstlisting}

\noindent The parent of this node in the tree. This is only \texttt{nullptr} for
the root.

\begin{lstlisting}
Node<T> *Node<T>::child[8]
\end{lstlisting}

\noindent The children of this node in the tree. These might be
\texttt{nullptr}.

\begin{lstlisting}
DAGInfo Node<T>::dag
\end{lstlisting}

\noindent A \texttt{DAGInfo} object providing a handle to the relevant portions
of the DAG for this node.

\begin{lstlisting}
bool Node<T>::is_leaf() const
\end{lstlisting}

\noindent A utility predicate answering the question: Is this node a leaf?


\section{Built-in distribution policies}

The distribution policy controls on which localities the nodes of the DAG
are ultimately placed. Each Method is required to define a distribution policy
to be used with that method (see below). DASHMM has a few built-in distribution
policies that can be used in user-defined methods. Additionally, there is a
default policy defined that represents the current best all-around policy
available in DASHMM. To make use of this default in a user-defined Method,
one merely has to include the following in the class definition:

\begin{lstlisting}[frame=]
using distropolicy_t = dashmm::DefaultDistributionPolicy;
\end{lstlisting}

\noindent \texttt{DefaultDistributionPolicy} is defined in
\texttt{dashmm/defaultpolicy.h}.

Each distribution policy only sets the locality of nodes that are not
automatically set by DASHMM. DAG nodes that have an automatically determined
locality include: the source DAG nodes, the target DAG nodes, and those
normal expansions on the source or target side representing the same node of
the source or target tree as a source or target DAG node.

The following distribution policies are included with DASHMM:

\begin{lstlisting}
SingleLocality
\end{lstlisting}

\noindent This distribution policy places all of the DAG nodes on a single
locality in the system. The locality can be selected in the constructor, and
has a default value of 0: \texttt{SingleLocality(int loc = 0)}.

\begin{lstlisting}
RandomDistro
\end{lstlisting}

\noindent This distribution places DAG nodes around the available localities at
random. Note that this is generally speaking a bad idea, but the scaling of
this distribution is not bad, even if the raw performance is bad. The seed for
the RNG can be set using the constructor, which has a default value:
\texttt{RandomDistro(int seed = 137)}.

\begin{lstlisting}
BHDistro
\end{lstlisting}

\noindent This distribution policy, which is the policy used for the
\texttt{BH} method, is compatible with any method; there is no dependence on
its operation on the sorts of DAG nodes or the operations performed on the DAG
edges. This policy works by starting at the target DAG nodes and working
backwards through the rest of the DAG. The locality of a node being examined is
set to the locality which minimizes the communication with other localities.
In deciding what is the minimal communication, the weight of the DAG edges is
used to approximate the cost of the message.

\begin{lstlisting}
FMM97Distro
\end{lstlisting}

\noindent This distribution policy only applies to the \texttt{FMM97} method.
It explores the source and target tree structure within the DAG. The normal
and intermediate DAG nodes associated with a node in the source tree, and
the normal DAG node associated with a node in the target tree, are placed on
the locality that contains the descendant source or target DAG nodes. To
determine the placement of the intermediate DAG node associated with a node
in the target tree, the policy considers the weight of the DAG edges to
minimize communication cost, and the color of the DAG edges to increase slack
time to hide communication latency.


\section{User-defined Expansions}

To define an expansion, a user need only create a class that conforms to the
following interface. For more details on the relation of Expansions to the
mathematics, please see section~\ref{sec:advconcepts} above. To implement a
user-defined Expansion, one must be familiar with the following DASHMM
constructs: \texttt{Point}, \texttt{ExpansionRole}, \texttt{ViewSet},
\texttt{Index}, \texttt{Operation} and \texttt{dcomplex\_t}.

In the following, we shall take the name of the user-defined Expansion to
be \texttt{Expansion}, but one can use any other name that one wishes.

Expansions are template types with two parameters, the Source and Target
types. The following would be the declaration of the user-defined Expansion:

\begin{lstlisting}[frame=]
template <typename Source, typename Target>
class Expansion;
\end{lstlisting}

\noindent It is not required, but the following aliases will be assumed to be
have been defined in the following description of the Expansion interface:

\begin{lstlisting}[frame=]
using source_t = Source;
using target_t = Target;
using expansion_t = Expansion<Source, Target>;
\end{lstlisting}

\noindent The latter introduces some brevity, while the former are mostly
for completeness.

\begin{lstlisting}
Expansion::Expansion(
    ExpansionRole role,
    double scale = 1.0,
    Point center = Point{})
\end{lstlisting}

\noindent This constructor creates the expansion object with the given
\texttt{center}, \texttt{scale} and \texttt{role}. There is no obligation on
the part of the expansion to use these inputs, but they will be provided for
those expansions that will need these data.
The scale provided to this constructor will be produced by
\texttt{Expansion::compute\_scale}.

\begin{lstlisting}
Expansion::Expansion(const ViewSet &views)
\end{lstlisting}

\noindent This constructor interprets the provided \texttt{views} as the data
serving this expansion. Unlike the previous constructor, this constructor
produces an object that does not own any data. Instead, it interprets existing
data.

Further, this constructor must be able to operate in a mode where \texttt{views}
is empty. This `shallow' mode of construction is for cases where the
expansion data is not needed, such as \texttt{StoT}, but where
some of the kernel parameters might be needed.

\begin{lstlisting}
Expansion::~Expansion()
\end{lstlisting}

\noindent The destructor should free the data of the expansion. In some
instances, the expansion will not own any data, and so this should do nothing.
Only if the expansion is \texttt{valid()} should this ever delete allocated
memory.

\begin{lstlisting}
void Expansion::release()
\end{lstlisting}

\noindent This will release the internal data for an expansion. These objects
need to support the ability to export the data making the expansion, losing
ownership of the data in the process. Further, expansion objects need to be
able to be constructed in a shallow way from existing data. \texttt{release()}
breaks the association. After \texttt{release()}, calls to \texttt{valid()}
must return false.

The simplest implementation of this is to have the object store a pointer to
memory allocated on the heap (as in \texttt{new char [size]}), and
\texttt{release()} can just set that pointer to \texttt{nullptr}.

\begin{lstlisting}
bool Expansion::valid(const ViewSet &views) const
\end{lstlisting}

\noindent Returns if the indicated views are valid. An expansion is valid if it
has data associated with it. If \texttt{views} is empty, this will check all
views.

\begin{lstlisting}
int Expansion::view_count() const
\end{lstlisting}

\noindent Return the current number of views for this object. This will either
be the full number for an object created with the first constructor, or a
smaller number for an expansion created by interpretation in the second
constructor.

\begin{lstlisting}
ViewSet Expansion::get_all_views() const
\end{lstlisting}

\noindent Get all current views of this object.

\begin{lstlisting}
ExpansionRole Expansion::role() const
\end{lstlisting}

\noindent Return the expansion role of this object.

\begin{lstlisting}
Point Expansion::center() const
\end{lstlisting}

\noindent Return the point around which the expansion is defined.

\begin{lstlisting}
size_t Expansion::view_size(int view) const
\end{lstlisting}

\noindent Return the view size for the specified \texttt{view}. This returns
the number of terms in the expansion for the given view. Do not confuse this
with the size in bytes of the data in a given view, which can be obtained with
\texttt{get\_all\_views()}.

\begin{lstlisting}
dcomplex_t Expansion::view_term(int view, size_t i) const
\end{lstlisting}

\noindent Get term \texttt{i} of the given \texttt{view}. The term is returned
as a complex number, so real-valued expansions must return a complex number.

\begin{lstlisting}
std::unique_ptr<expansion_t>
Expansion::S_to_M(Point center, source_t *first, source_t *last) const
\end{lstlisting}

\noindent Create a multipole expansion for a given set of sources. The
expansion will have the given \texttt{center}, and the sources are gives as
pointers to the \texttt{first} and one past the \texttt{last} record.
The returned expansion will have a role of \texttt{kSourcePrimary}.

\begin{lstlisting}
std::unique_ptr<expansion_t>
Expansion::S_to_L(Point center, source_t *first, source_t *last) const
\end{lstlisting}

\noindent Create a local expansion for a given set of sources. The expansion
will have the given \texttt{center}, and the sources are gives as pointers to
the \texttt{first} and one past the \texttt{last} record.
The returned expansion will have a role of \texttt{kTargetPrimary}.

\begin{lstlisting}
std::unique_ptr<expansion_t>
Expansion::M_to_M(int from_child) const
\end{lstlisting}

\noindent Change the center of a multipole expansion. The shift in the center
is specified through \texttt{from\_child}, which indicates the child from which
the expansion is being converted.
This expansion will have a role of \texttt{kSourcePrimary}.
The returned expansion will have a role of \texttt{kSourcePrimary}.

\begin{lstlisting}
std::unique_ptr<expansion_t>
Expansion::M_to_L(Index s_index, Index t_index) const
\end{lstlisting}

\noindent Convert a multipole expansion to a local expansion. To specify the
change the source and target indices (\texttt{s\_index} and \texttt{t\_index}),
are provided.
This expansion will have a role of \texttt{kSourcePrimary}.
The returned expansion will have a role of \texttt{kTargetPrimary}.

\begin{lstlisting}
std::unique_ptr<expansion_t>
Expansion::L_to_L(int to_child) const
\end{lstlisting}

\noindent Convert a local expansion to a local expansion for a child. The
child is specified by \texttt{to\_child}, which gives the child of the tree
node associated with this expansion that the resulting expansion should be
associated with.
This expansion will have a role of \texttt{kTargetPrimary}.
The returned expansion will have a role of \texttt{kTargetPrimary}.

\begin{lstlisting}
void Expansion::M_to_T(target_t *first, target_t *last) const
\end{lstlisting}

\noindent Apply the effect of a multipole expansion to a set of targets,
specified by a pointer to the \texttt{first} and one past the \texttt{last}
target.
This expansion will have a role of \texttt{kSourcePrimary}.

\begin{lstlisting}
void Expansion::L_to_T(target_t *first, target_t *last) const
\end{lstlisting}

\noindent Apply the effect of a local expansion to a set of targets, specified
by a pointer to the \texttt{first} and one past the \texttt{last} target.
This expansion will have a role of \texttt{kTargetPrimary}.

\begin{lstlisting}
void Expansion::S_to_T(source_t *s_first, source_t *s_last,
                       target_t *t_first, target_t *t_last) const
\end{lstlisting}

\noindent Apply the direct interaction of a set of sources to a set of targets.
The sources and targets are specified by pointers to the first and one past the
last record.

\begin{lstlisting}
std::unique_ptr<expansion_t> Expansion::M_to_I() const
\end{lstlisting}

\noindent Create an intermediate expansion from a multipole expansion.
This expansion will have a role of \texttt{kSourcePrimary}.
The returned expansion will have a role of \texttt{kSourceIntermediate}.

\begin{lstlisting}
std::unique_ptr<expansion_t>
Expansion::I_to_I(Index s_index, Index t_index) const
\end{lstlisting}

\noindent Translate a source-side intermediate expansion into a target-side
intermediate expansion. The \texttt{s\_index} and \texttt{t\_index} are the
index of the tree nodes that are represented by this object, and the
resulting expansion, respectively.
This expansion will have a role of \texttt{kSourceIntermediate}.
The returned expansion will have a role of \texttt{kTargetIntermediate}.

\begin{lstlisting}
std::unique_ptr<expansion_t>
Expansion::I_to_L(Index t_index) const
\end{lstlisting}

\noindent Translate a target-side intermediate expansion into a local
expansion. The target tree node's index, \texttt{t\_index}, is provided.
This expansion will have a role of \texttt{kTargetIntermediate}.
The returned expansion will have a role of \texttt{kTargetPrimary}.

\begin{lstlisting}
void Expansion::add_expansion(const expansion_t *temp)
\end{lstlisting}

\noindent Add the given expansion to this expansion. Typically this involves
summing the coefficients, but can be more involved in some cases.

\begin{lstlisting}
static void Expansion::update_table(
    int n_digits,
    double domain_size,
    const std::vector<double> &kernel_params)
\end{lstlisting}

\noindent Update a kernel table. This should generate or update a kernel table
associated
with this expansion type. The kernel table is a mechanism for precomputing
values that are required by expansion operations. As the same table should
serve every instance of this class, the table should be a static member of the
class. This routine will either allocate and fill these tables (if this is the
first call to this function during the program lifetime, or if the table has
been explicitly deleted) or it will replace the values in the table without
necessarily reallocating the memory for the table (if this is a subsequent call
to this function).

The value of \texttt{n\_digits} and \texttt{kernel\_params} are ultimately
provided by the call to \texttt{evaluate()}. The
\texttt{domain\_size} is computed by DASHMM after the tree is constructed.

\begin{lstlisting}
static void Expansion::delete_table()
\end{lstlisting}

\noindent Destroy the kernel table. This should destroy any tables that exist
that are associated with this type of expansion.

\begin{lstlisting}
static double Expansion::compute_scale(Index index)
\end{lstlisting}

\noindent Compute the scale to pass into expansion constructors. The will only
be called
after the table exists, so the implementation can rely on the existence of the
table. In particular, kernel parameters needed by this routine should be
stored in the table.

\begin{lstlisting}
static int Expansion::weight_estimate(
    Operation op,
    Index s = Index{},
    Index t = Index{})
\end{lstlisting}

\noindent Compute an estimate of the cost to send the result of the given
operation
across the network. In some expansions, the operation is enough to determine
the cost. In others, the source and target indices, \texttt{s} and \texttt{t}.



\section{User-defined Methods}

To create a user-defined Method, one must create a class that conforms
to the following interface. For a description of the four primary routines
of a Method, please see~\ref{sec:advconcepts}. To implement a user-defined
Method, one must be familiar with the following DASHMM constructs: the
\texttt{Expansion} concept, nodes of the tree (both source and target),
\texttt{DomainGeometry} and \texttt{DAGInfo}.

In the following we shall take the name of the user-defined Method to be
\texttt{Method}, but one can use any name that one wishes.

Methods are templates over three parameters, the Source, Target and Expansion
types. The following is the full declaration of a user-defined Method:

\begin{lstlisting}[frame=]
template <typename Source, typename Target,
          template <typename, typename> class Expansion>
class Method;
\end{lstlisting}

\noindent It is helpful to also define the following aliases as members of
any Method:

\begin{lstlisting}[frame=]
using source_t = Source;
using target_t = Target;
using expansion_t = Expansion<Source, Target>;
using method_t = Method<Source, Target, Expansion>;
using sourcenode_t = Node<Source>;
using targetnode_t = Node<Target>;
\end{lstlisting}

\noindent The above aliases will be used in the following description.

\begin{lstlisting}
Method::distropolicy_t
\end{lstlisting}

\noindent
Any method must define this type to specify the distribution policy that is to
be used with the method. This allows for the possibility that special details
of the implemented \texttt{Method} might allow for better distribution of the
DAG. If there is no such better distribution, or the implementer is not
concerned, or if the default provided by DASHMM is found to be sufficient, then
one can use \texttt{dashmm::DefaultDistributionPolicy} defined in
\texttt{dashmm/defaultpolicy.h}.

\begin{lstlisting}
Method::Method()
\end{lstlisting}

\noindent
Methods must have a default constructor. Note that other constructors are
allowed. Methods instances used for particular evaluations are copied so
methods can support runtime parameters.

\begin{lstlisting}
void Method::generate(sourcenode_t *curr, DomainGeometry *domain) const
\end{lstlisting}

\noindent
This routine is called at the leaves of the source tree to generate the initial
expansions from the source data. The leaf node in question is provided via
\texttt{curr}. For use in cases where the size of the node is relevant,
the problem \texttt{domain} is also provided. The typical interaction with
\texttt{curr} is to use its \texttt{dag} member.

\texttt{generate} is responsible for creating the DAG nodes that the method
requires for the given source tree leaf node. Further, any operations between
these nodes should be scheduled during \texttt{generate} (via the many
\texttt{DAGInfo} methods). The typical operation scheduled during
\texttt{generate} is \texttt{StoM}.


\begin{lstlisting}
void Method::aggregate(sourcenode_t *curr, DomainGeometry *domain) const
\end{lstlisting}

\noindent
This routine is called for the internal nodes of the source tree to generate
any expansions needed by the method that are associated with non-leaf source
tree nodes. The current node, \texttt{curr}, and the overall \texttt{domain}
are provided.

\texttt{aggregate} is responsible for creating the DAG nodes that the method
requires. Further, operations between nodes should be scheduled. The typical
operation for \texttt{aggregate} is \texttt{MtoM}.

\begin{lstlisting}
void Method::inherit(
    targetnode_t *curr,
    DomainGeometry *domain,
    bool curr_is_leaf) const
\end{lstlisting}

\noindent
In \texttt{inherit} information from parents in the target tree are propagated
to children. In addition to the node, \texttt{curr} and the \texttt{domain},
whether is the current node is a leaf if indicated with
\texttt{curr\_is\_leaf}. This is necessary as some methods might allow for
internal target tree nodes to nonetheless act as if they were leaves.

\texttt{inherit} is responsible for creating the DAG nodes that the method
requires, and for scheduling the operations between DAG nodes. The typical
operation for \texttt{inherit} is \texttt{LtoL}.


\begin{lstlisting}
void Method::process(
    targetnode_t *curr,
    std::vector<sourcenode_t *> &consider,
    bool curr_is_leaf,
    DomainGeometry *domain) const
\end{lstlisting}

\noindent
The most complex work of the method often takes place in \texttt{process}. In
addition to the current node, the domain geometry, and if this leaf is a node,
a vector of source tree nodes are provided. This vector, \texttt{consider},
gives the list of source nodes that might have an impact on the calculation of
the potential for the target locations represented by this node.

\texttt{consider} should be examined, and where possible elements should be
removed from the vector when their effect on the targets in \texttt{curr} can
be computed. That is, if some operation is scheduled between a node in
\texttt{consider} and \texttt{curr}, then that source node has been handled,
and can be removed. In some instances, a node might be removed from
\texttt{consider} and the children of that node will be added to consider.

At the end of \texttt{process}, the set of nodes in \texttt{consider} will be
different than were passed into this function. The resulting \texttt{consider}
will be passed to the children of \texttt{curr}, unless \texttt{curr\_is\_leaf}
was true. In that case, it should be that \texttt{consider} will have been
completely used by the call to \texttt{process}.

\begin{lstlisting}
void Method::refine_test(
    bool same_sources_and_targets,
    const targetnode_t *curr,
    const std::vector<sourcenode_t *> &consider) const
\end{lstlisting}

\noindent
This method determines if the given target tree node should be refined, or if
it represents a leaf of the DAG. This decision is based on \texttt{curr},
\texttt{consider} and if the sources and targets are identical in this
evaluation. The latter is provided as \texttt{same\_sources\_and\_targets}.

This method should return \texttt{true} if the target tree should be refined
further, and \texttt{false} otherwise.



\section{User-defined distribution policies}

For users defining their own Methods, it can potentially be helpful to also
define a distribution policy for that method if the built-in methods cannot
take advantage of details of the generated DAG. This section outlines the
concept of \texttt{DistributionPolicy}, giving details on the required
interface. Distribution policies have a relatively small interface. For the
sake of discussion, we shall call the user-defined policy \texttt{Policy},
but one can give any name to the policy.

\begin{lstlisting}
Policy::Policy()
\end{lstlisting}

\noindent
Distribution policies require either a default constructor, or a constructor
with all arguments given default values.

\begin{lstlisting}
void Policy::compute_distribution(DAG &dag)
\end{lstlisting}

\noindent
This is the main distribution method. After the DAG is fully discovered, it will
be passed into this function where the localities of every DAG node need to be
given a value. Because DASHMM sets a few localities in a fixed way, only those
DAG nodes with a locality set to \texttt{-1} (the default value) should have
a locality assigned. Changing the locality of a DAG node that has already had
a locality assigned will lead to program failure and termination.

Other than the previous two conditions, there are no other restriction on the
implementation. However, it is generally best to avoid very deep recursion.
HPX-5 threads each have a stack associated with them, and these stacks are
generally small, given that there will be a large number of threads. Thus, it
is much easier to overrun a thread's stack.

\begin{lstlisting}
void Policy::assign_for_source(DAGInfo &dag, int locality)
\end{lstlisting}

\noindent
The routine is called for every source tree node during DAG discovery. It is an
opportunity to set localities when the DAG nodes are created, and not all at
once in \texttt{compute\_distribution}.

WARNING: There is inadequate contention management available to safely use this
routine without a thorough understanding of the internal workings of DASHMM.
It is strongly advised to not give this method a non-trivial implementation.

NOTE: This is an experimental feature, and should not be considered to be part
of the DASHMM API; it may be removed in the future.

\begin{lstlisting}
void Policy::assign_for_target(DAGInfo &dag, int locality)
\end{lstlisting}

\noindent
The routine is called for every target tree node during DAG discovery. It is an
opportunity to set localities when the DAG nodes are created, and not all at
once in \texttt{compute\_distribution}.

WARNING: There is inadequate contention management available to safely use this
routine without a thorough understanding of the internal workings of DASHMM.
It is strongly advised to not give this method a non-trivial implementation.

NOTE: This is an experimental feature, and should not be considered to be part
of the DASHMM API; it may be remove in the future.
