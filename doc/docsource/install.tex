\chapter{Installing DASHMM}
\label{ch:install}

This chapter outlines what is needed to install and use of DASHMM, and
gives a brief sketch of the demo programs included with DASHMM.

\section{Prerequisites}

DASHMM v.\ 1.0.0 depends on one external library: HPX-5. The current
version of DASHMM depends on version 4.0.0 of HPX-5 or later, which
may be found at \url{https://hpx.crest.iu.edu/}. Please see the
official HPX-5 documentation for instructions on how to build, install
and run HPX-5 on your system.

The DASHMM build system relies on the {\tt pkg-config} utility to
specify the needed HPX-5 compilation and linking options, so it is
important to add the correct path for HPX-5 to your {\tt
  PKG\_CONFIG\_PATH} environment variable. For example, assuming HPX-5
is installed in {\tt /path/to/hpx}, this can be accomplished using
{\tt bash} with:

\begin{lstlisting}[language=bash, frame=]
export PKG_CONFIG_PATH=/path/to/hpx/lib/pkgconfig:$PKG_CONFIG_PATH
\end{lstlisting}

\section{Building DASHMM}

The DASHMM library is straightforward to build. Once the previous
prerequisite is met, one needs to perform the following steps:

\begin{enumerate}
\item Unpack the source code into some convenient directory. For the
  sake of discussion, this guide assumes that the code has been
  unpacked in {\tt /path/to/dashmm/}.
\item In {\tt /path/to/dashmm/} can be found {\tt Makefile}. There are
  few (if any) changes that need to be made to this file for
  successful compilation. The most likely change would be to modify
  the compiler used. Any compiler supported by HPX-5 will be able to
  compile DASHMM, provided it also support the C++11 standard. For
  example, to change the compiler from the default ({\tt g++}) to the
  Intel compiler, one needs to replace {\tt CXX=g++} with {\tt
    CXX=icpc}.
\item Run {\tt make} from {\tt /path/to/dashmm}. This should build the
  library statically and it will be ready to link with your specific
  application.
\end{enumerate}

Because DASHMM is heavily templated, a good deal of the code is
contained in the header files. This means that much of the compilation
of the DASHMM code will occur when the user code is compiled. This can
increase the compilation time of user code, but the added flexibility
of DASHMM is worth the minor increase in compilation time.

\section{Linking against DASHMM}

To build a program using the DASHMM library, only a few things need to
be done. DASHMM builds in place, so when compiling code that used the
library, one must specify where to look for the header files, and
where to look for the built library. Further, because DASHMM relies on
HPX-5, one must also specify how to find HPX-5. For HPX-5 this is the
easiest with the {\tt pkg-config} utility.

Assuming that DASHMM was built in {\tt /path/to/dashmm}, to compile
code (with, for example {\tt g++}) one must specify the following
arguments for compilation:

\begin{lstlisting}[language=bash, frame=]
-I/path/to/dashmm/include $(shell pkg-config --cflags hpx)
\end{lstlisting}

\noindent Similarly, one must specify the following arguments for linking:

\begin{lstlisting}[language=bash, frame=]
-L/path/to/dashmm/lib -ldashmm $(shell pkg-config --libs hpx)
\end{lstlisting}

\noindent Examples of this can be found in the demo programs included with
DASHMM.

\section{DASHMM demo programs}

Included with DASHMM are several test codes that demonstrate the use
of the library. These are found in the {\tt /path/to/dashmm/demo}
subfolder. Each test can be built simply by running {\tt make} in the
code's directory. Detailed information about each example can be found in
the provided {\tt README} file.

\subsection{Basic demo}
The {\tt basic} demo code creates a random distribution of source and
target points and computes the potential at the targets due to the
sources using any of the built-in kernels provided with DASHMM. A user
can request a summary of the options to the test code by running the
code with {\tt --help} as a command line argument, or by reading {\tt
  /path/to/dashmm/demo/basic/README}.

\subsection{Time-stepping demo}
The {\tt stepping} demo code creates a distribution of particles and
computes their acceleration and integrates their motion forward in
time. Note that the point of the demo is not to provide a great time
integrator, so many fine points of creating a good integrator are
skipped. Instead, {\tt stepping} demonstrates those features of DASHMM
that enable time-stepping codes as a use-case for DASHMM. A user can
request a summary of the options to the code by running the code with
{\tt --help} as a command line argument, or by reading {\tt
  /path/to/dashmm/demo/stepping/README}.

\subsection{User-defined expansion demo}

The {\tt user} demo provides a skeleton code that implements a new
expansion type. The code is documented, and the requirements of the
Expansion concept are outlined in the in-source comments. This example
should be considered to be more advanced, and would require
investigation of the advanced interface to understand completely.
