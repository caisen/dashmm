\chapter{Introduction to DASHMM}
\label{ch:intro}

The {\it Dynamic Adaptive System for Hierarchical Multipole Methods}
(DASHMM) is a C++ library providing a general framework for
computations using multipole methods. In addition to the flexibility
to handle user-specified methods and expansions, DASHMM includes
built-in methods and expansions, including the Barnes-Hut~(BH) and
Fast Multipole Method~(FMM), and expansions implementing the Laplace
and Yukawa kernels.

DASHMM is designed to make its adoption and use as easy as possible,
and so the interface to DASHMM provides both an easy-to-use basic
interface (see Chapter~\ref{ch:basic}) and a more advanced interface
(see Chapter~\ref{ch:advanced}). The basic interface allows a
user to get up and running with multipole methods as quickly as
possible. However, more advanced use-cases, especially for users that
are implementing their own methods and expansions, will need to
explore the advanced interface.

DASHMM is built using the advanced runtime system, HPX-5, but the
basic, and much of the advanced interface does not require any
knowledge of how to use HPX-5 directly. Instead, DASHMM insulates the
users from the specific details of HPX-5, allowing expression of the
multipole method application in higher level concepts than the
specifics of threads and execution control structures. Nevertheless,
it can be helpful to have a sense of the conceptual underpinnings of
HPX-5, and how those relate to DASHMM. Information about this can be
found in this guide in both the basic and advanced interface. Even
more information can be found at the HPX-5 website
(\url{https://hpx.crest.iu.edu/}).

For a description of the techniques that go into the parallel
execution of DASHMM, and for a discussion of the conceptual framework,
please see the code paper: {\it ``DASHMM: Dynamic Adaptive System for
  Hierarchical Multipole Methods''} in Communications in Computational
Physics, Vol. 40 (2016), No. 4, pp. 1106-1126.

This document covers version 1.0.0 of DASHMM. For the latest news and
updates of DASHMM, please visit the DASHMM website:
\url{https://www.crest.iu.edu/projects/dashmm/}.

The DASHMM project has adopted semantic versioning (\url{http://semver.org}).
Chapters \ref{ch:basic} and \ref{ch:advanced} should be considered to be the
specification of the interface to the library.

In the following, snippets of code or the names of code constructs
will be set in a fixed width font. For example, {\tt main()}. Unless
otherwise indicated, every construct presented in this guide is a
member of the {\tt dashmm} namespace.